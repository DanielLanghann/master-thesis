%*************************************************************************
% Chapter 1
%*************************************************************************


\chapter{Einleitung}
\label{ch:intro}
Lorem ipsum at nusquam appellantur his, labitur bonorum pri no \citep{dueck:trio}. His no decore nemore graecis. In eos meis nominavi, liber soluta vim cu. Sea commune suavitate interpretaris eu, vix eu libris efficiantur.

%
% Section: Motiva
%
\section{Motivation}
\label{sec:intro:motivation}
\graffito{Note: The content of this chapter is just some dummy text. It is not a real language.}
Illo principalmente su nos. Non message \emph{occidental} angloromanic da. Debitas effortio simplificate sia se, auxiliar summarios da que, se avantiate publicationes via. Pan in terra summarios, capital interlingua se que. Al via multo esser specimen, campo responder que da. Le usate medical addresses pro, europa origine sanctificate nos se.

Cras , leo ac adipiscing adipiscing, erat justo vulputate arcu, non sollicitudin ipsum dolor eget lectus. Nulla sed mi non ipsum varius consequat sit amet nec ipsum. Donec ac elit id nibh pretium pulvinar non ut ipsum. Integer congue iaculis augue ac porttitor. Suspendisse sed enim ac eros hendrerit adipiscing. Integer elit libero, lacinia vitae pharetra a, ullamcorper vitae metus. In tempor, est id imperdiet pulvinar, tellus nibh lacinia diam, a eleifend dui lectus non turpis.

%
% Section: Ziele
%
\section{Ziel der Arbeit}
\label{sec:intro:goal}
Errem omnium ea per, pro \ac{UML} congue populo ornatus cu, ex qui dicant nemore melius. No pri diam iriure euismod. Graecis eleifend appellantur quo id. Id corpora inimicus nam, facer nonummy ne pro, kasd repudiandae ei mei. Mea menandri mediocrem dissentiet cu, ex nominati imperdiet nec, sea odio duis vocent ei. Tempor everti appareat cu ius, ridens audiam an qui, aliquid admodum conceptam ne qui. Vis ea melius nostrum, mel alienum ac elit id nibh pretium pulvina euripidis eu.

%
% Section:  der Arbeit
%
\section{Gliederung}
\label{sec:intro:structure}
nulla fastidii ea ius, exerci suscipit instructior te nam, in ullum postulant quo. Congue quaestio philosophia his at, sea odio autem vulputate ex. Cu usu mucius iisque voluptua. Sit maiorum propriae at, ea cum \ac{API} primis intellegat. Hinc cotidieque reprehendunt eu nec. Autem timeam deleniti usu id, in nec nibh altera.



%*************************************************************************
% Chapter 2
%*************************************************************************


\chapter{Terminologische Grundlagen}

%*************************************************************************
% Chapter 3
%*************************************************************************

\chapter{Initial Zuordnung von Datasets zu Datastores}


%*************************************************************************
% Chapter 4
%*************************************************************************

\chapter{Dynamische Allokation und Re-Allokation von Datasets zu Datastores}


%*************************************************************************
% Chapter 5
%*************************************************************************

\chapter{Datenmodell und Datenbankseitige Umsetzung von Datastore Re-Allokationen}

%*************************************************************************
% Chapter 6
%*************************************************************************

\chapter{Schlussbetrachtung}