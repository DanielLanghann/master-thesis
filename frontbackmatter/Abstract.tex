%*******************************************************
% Abstract in English
%*******************************************************
\pdfbookmark[0]{Abstract}{Abstract}


\begin{otherlanguage}{american}
	\chapter*{Abstract}
	In der vorliegenden Masterarbeit wird das Thema Dynamische Allokation von Datasets zu 
	Datastores behandlet.
	Die Motivation für die Arbeit ist es, einen Vorschlag zu unterbreiten, wie in einem Umfeld aus 
	heterogenen Datasets, für die mehr als ein Datenbank-System ideal ist, eine
	dynamische Zuweisung und Neuzuweisung von Datasets zu Datastores zu realisieren.
	In der Vergangenheit wurde bereits viel an dem Themenbereich Polystores geforscht.
	Polystores beinhalten mindestens zwei verschiedenartige Datastores also letztendlich
	Datenbank-Systeme um hetoregene Datasets, zum Beispiel transaktionsbasierte Daten auf der
	einen Seite und aggregierte Daten für die Analyse von zum Beispiel sogenannten Key Perfomance
	Indikatoren (KPI) auf der anderen Seite optimal verarbeiten zu können.
	Eine bislang ungelöste Herausforderung in einem polystoren Umfeld ist, die Reaktion auf sich 
	verändernde Workloads in der Applikaton bzw. im Gesamtsystem. Wie reagiert man bezogen auf das oben genannte Beispiel, wenn in dem 
	transaktionsbasierten Bereich des Gesamtsystems vermehrt analytische 
	Abfragen entstehen, also GET Requests mit langen und sehr langen Laufzeiten. Dann wäre 
	es wünschenswert, dass diese Abfragen zukünftig über den Datastore, der sich auf 
	analytische Abfragen fokussiert, umgeleitet werden. \newline

	\noindent
	Genau hier setzt die vorligende Arbeit an. Es soll ein Vorschlag unterbreitet werden, wie man 
	in einem polystoren Systemumfeld, dynamisch auf sich verändernde Workloads reagieren kann.
	Der erste Schwerpunkt beschäftig sich damit, bezogen auf gegebene Anforderungen an 
	ein Gesamtsystem eine initiale Zuweisung von Datasets zu Datastores vorzuschlagen.
	Es wird ein Prototyp entwickelt, der algorithmisch eine entsprechende Zuweisung ermittelt.
	Der zweite Schwerpunkt fokussiert sich auf die permanante und dynamische Analyse von Workloads
	um bei sich verändernden Paramtern bezogen auf den initialen Zustand mit einer Re-Allokation
	der Datasets zu reagieren bzw. einen angepassten Vorschlag zu unterbreiten.  Für diesen 
	zweiten Schwerpunkt wird ebenfalls ein Prototyp entwickelt, der als Input die Analysedaten 
	des Workloads erhält, und beim Erreichen bestimmter Schwellwerte einen Vorschlag für eine Neuzuweisung
	unterbreitet. 
	Für die konkrete Ausführung der angepassten Zuweisungen wird ein theoretischer Vorschlag gemacht und diskutiert.\newline

	\noindent
	Die Arbeit schließt wiederum mit einer Zusammenfassung der Ergebnisse bezogen auf die Ausführung
	und Testung der Prototypen.
	Darüber hinaus sollen Vorschläge unterbereitet werden, an welchen Themen noch gearbeitet werden sollte,
	um tatsächlich ein komplett autonom arbeitendes Gesamtsystem auf der Basis von Polystores zu realisieren.


\end{otherlanguage}
